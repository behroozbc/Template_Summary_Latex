\documentclass[a4paper, 9pt, twocolumn]{extarticle}
\usepackage{amsmath,graphicx,amssymb,cite}
\usepackage[utf8]{inputenc}
\usepackage[english]{babel}

\usepackage{url}


\addtolength{\textwidth}{2.1cm}
\addtolength{\topmargin}{-2.4cm}
\addtolength{\oddsidemargin}{-1.1 cm}
\addtolength{\textheight}{4.5cm}
\setlength{\columnsep}{0.7cm}

% User defined macros
\def\x{{\mathbf x}}
\def\L{{\cal L}}
\def\SM{{\mathcal S}}
\def\SMO{{\mathcal S^{\mathrm{chroma}}}}
\def\SMS{{\mathcal S^{\mathrm{enh}}}}
\def\SMP{{\mathcal S^{\mathrm{path}}}}
\def\SMPI{{\mathcal S^{\mathrm{struct}}}}
%\def\SMPC{{\mathcal S^{\mathrm{pc}}}}
\def\SMPC{{\mathcal S^{\mathrm{pb}}}}

\pagestyle{empty}

\begin{document}

\date{\normalsize \today}

\title{\vspace{-8mm}\textbf{\Large
    Music Synchronization\footnote{This is the summary for the reading assignment,
      which is part of the exercise of the lecture \emph{Music Processing Analysis}, Winter Term 2024/25,
      Friedrich-Alexander Universit\"at Erlangen-N\"urnberg.
      Instructor: Prof.\ Dr.\ Meinard M\"uller,
      Tutor: Simon Schw\"ar.
    }}}

% Hier die Namen und Daten der beteiligten Autoren eintragen
\author{
  {
      \begin{minipage}{\textwidth}
        \center
        Behrooz bozorgchamy\\
        \small
        Friedrich-Alexander Universit\"at Erlangen-N\"urnberg
        \protect\\{} %
        \url{behrooz.bozorgchamy@fau.de}
      \end{minipage}
    }
}

\maketitle
\thispagestyle{empty}
The following summary is based on Chapter 3 Music Synchronization from FMP book\cite{Mueller21_FMP_SPRINGER}.
%%%%%%%%%%%%%%%%%%%%%%%%%%%%%%%%%%%%%%%%%%%%%%%%%%%%%%%%%%%%%%%%%%%%%%%%%%%%%%
\section{Introduction}
\label{section:introduction}
%%%%%%%%%%%%%%%%%%%%%%%%%%%%%%%%%%%%%%%%%%%%%%%%%%%%%%%%%%%%%%%%%%%%%%%%%%%%%%

This chapter focuses on the process of music synchronization, which
involves aligning different musical representations, such as audio
recordings and sheet music. The core concept is to extract meaningful
features from these representations and then employ a technique called
Dynamic Time Warping (DTW) to find the optimal temporal correspondence
between them. Let's start, explains some usefull definitions 
\textbf{equal-tempered scale},In the twelve-tone equal-tempered scale,
the octave is divided into twelve scale steps with equally spaced
fundamental frequencies on a logarithmic frequency axis.

\section{Audio Features}
\label{section:Audio Features}
The chapter emphasizes the importance of \textbf{chroma features} in music
synchronization. Chroma features essentially capture the distribution of
energy across the 12 pitch classes (C, C\#, D, ..., B) while
disregarding octave information. This characteristic makes them robust
to variations in timbre and dynamics, allowing for effective comparison
of different musical interpretations.
\subsection{Log-Frequency Spectrogram}
The process of extracting chroma features starts with a
\textbf{log-frequency spectrogram}. The \emph{Log-Frequency Spectrogram}
elucidates the transformation of an audio recording into a
representation that reveals the energy distribution across different
pitches. This is a crucial step in music synchronization as it makes
music data comparable and algorithmically accessible.
The process leverages the \textbf{logarithmic perception of frequency in
  human hearing} and aims to create a time-frequency representation where
the frequency axis is logarithmic and labeled by the pitches of the
equal-tempered scale.

Here\textquotesingle s a breakdown of the process:

\begin{enumerate}
  \def\labelenumi{\arabic{enumi}.}
  \item
        \textbf{Starting with a spectrogram:} The spectrogram, obtained using
        the Short-Time Fourier Transform (STFT), provides a visual
        representation of the frequency content of an audio signal over time.
  \item
        \textbf{Mapping to the Equal-Tempered Scale:} Each spectral
        coefficient in the spectrogram is assigned to a pitch in the
        equal-tempered scale. This is done by finding the pitch whose center
        frequency is closest to the frequency of the spectral coefficient.
  \item
        \textbf{Creating the Log-Frequency Spectrogram:} A log-frequency
        spectrogram is then generated by grouping the spectral coefficients
        according to their assigned pitches. For each pitch, the squared
        magnitudes of the corresponding spectral coefficients are summed up.
        This results in a representation where the frequency axis is now
        logarithmic and labeled by MIDI pitches.
\end{enumerate}
However, this process encounters challenges, particularly in
representing low pitches:
\begin{itemize}
  \item
        \textbf{Limited Frequency Resolution:} The STFT\textquotesingle s
        Hop size can be insufficient to accurately represent
        the lower pitches due to their narrower bandwidths in the logarithmic
        scale. This can lead to a poor representation of low-frequency
        content.
  \item
        \textbf{Trade-off Between Time and Frequency Resolution:} Increasing
        the window length in the STFT can improve frequency resolution but
        leads to a decrease in temporal resolution, potentially losing crucial
        information about note onsets.
\end{itemize}
A solution to these challenges is a \textbf{multiresolution approach}.
This involves using multiple spectrograms computed with different
sampling rates and window sizes. A higher sampling rate and shorter
window are used for high pitches to prioritize temporal resolution,
while a lower sampling rate and longer window are used for low pitches
to prioritize frequency resolution.

From this log-frequency spectrogram, the \textbf{chromagram} is derived
by summing up all pitch coefficients that belong to the same chroma. The
\emph{chroma features} represent the harmonic content of music. This
representation is fundamental to music synchronization as it captures
the essential harmonic information while remaining robust to variations
in timbre and instrumentation.
\subsection{Chroma features}
Chroma features are derived from the log-frequency spectrogram
and essentially aggregate all spectral information related to a
particular pitch class into a single coefficient.

The process involves the \textbf{Pitch Class Aggregation,} Which For
each of the 12 chroma values (C, C\#, D, ..., B), all the pitch
coefficients in the log-frequency spectrogram belonging to the same
chroma are summed up. This means that all notes that are an octave apart
(like C1, C2, C3) contribute to the same chroma band.

This results in a \textbf{chromagram}, a time-chroma representation
where each time frame is characterized by a 12-dimensional vector
representing the energy distribution across the 12 chroma values.

A chromagram can be generated from a musical score by extracting
the notes, mapping each to one of the 12 pitch classes (ignoring octaves),
accumulating the duration or intensity of each pitch class
throughout the score, and then creating a visual representation
where the pitch classes are on the x-axis and the accumulated
values on the y-axis. 
This results in a 12-dimensional vector for each time frame, providing a snapshot of the harmonic content.
The advantages of using chroma features is \textbf{Chroma features
aggregate all the spectral information relating to a given pitch class into a single coefficient},
\textbf{Chroma features can be normalized to be invariant to
changes in dynamics} and
\textbf{Chroma features are well-suited to characterize the
melodic and harmonic progression of music}.
% The highlights the following \textbf{advantages of using chroma
% features}:

% \begin{itemize}
% \item
%   \textbf{
%     Chroma features aggregate all the spectral information relating to a given pitch class into a single coefficient.}
% \item
%   \textbf{Chroma features can be normalized to be invariant to changes in dynamics.} 
% \item \textbf{Chroma features are well-suited to characterize the melodic and harmonic progression of music}
% \end{itemize}

However, this reduction in dimensionality also results in some loss of
information. For instance, notes with the same chroma but in different
octaves become indistinguishable in the chromagram.

Some technics which can help us to not lose information:

\begin{itemize}
  \item
        \textbf{Logarithmic Compression:} Applying logarithmic compression to
        the chromagram reduces the dynamic range of the features, emphasizing
        weaker but potentially significant harmonic components that might
        otherwise be overshadowed by stronger components.
  \item
        \textbf{Normalization:} Normalizing the chroma vectors ensures that
        they are invariant to changes in dynamics, making comparisons between
        different performances or sections of music more meaningful.
\end{itemize}

Furthermore, the \textbf{cyclic nature of chroma features} is explaining
how a cyclic shift of the chroma vector can simulate transpositions of
music. It concludes by emphasizing that the choice of specific chroma
feature variants and processing techniques depends heavily on the
intended application.

\section{Dynamic Time Warping (DTW)}
Dynamic Time Warping (DTW) is then introduced as a technique
for aligning two sequences of chroma features.

\subsection{ Basic Approach}
DTW aims to find an optimal alignment that minimizes the total cost of matching
corresponding elements in the sequences, effectively compensating for
differences in tempo between different musical interpretations.

A \textbf{warping path} represents the alignment between the two
sequences, defined as a sequence of index pairs that satisfy specific
conditions:

\begin{itemize}
  \item
        \textbf{Boundary condition}: The path starts at the beginning of both
        sequences and ends at the end of both sequences.
  \item
        \textbf{Monotonicity condition}: The path progresses monotonically
        through both sequences, reflecting faithful timing.
  \item
        \textbf{Step size condition}: The path only allows for specific
        movements between cells of the cost matrix, ensuring a continuous
        alignment.
\end{itemize}
The cosine distance or local cost measure, c, is a function that quantifies the dissimilarity between two elements in a feature space. we select the cosine distance because it measures the difference in energy distributions across chroma bands while being invariant to changes in dynamics.
The \textbf{total cost} of a warping path is calculated by summing up
the local cost of aligning each pair of elements connected by the path.
An \textbf{optimal warping path} is the one with the minimal total cost
among all possible paths, and the \textbf{DTW distance} between two
sequences is defined as the total cost of this optimal path.

Moreover, A \textbf{dynamic programming algorithm} for efficiently
computing the DTW distance and the corresponding optimal warping path.
The algorithm operates by recursively building an accumulated cost
matrix, where each entry represents the minimum cost of aligning
prefixes of the two input sequences.
\subsection{DTW variants}
Several DTW variants are presented, each offering specific
controls and optimizations for the alignment process:
\begin{itemize}
  \item
        \textbf{Modified Step Size Condition}: By adjusting the allowed steps
        in the warping path, it is possible to constrain the local slope of
        the alignment, preventing overly stretched or compressed sections.
  \item
        \textbf{Local Weights}: Different weights can be assigned to vertical,
        horizontal, and diagonal steps in the warping path, allowing for
        preferences in the alignment direction.
  \item
        \textbf{Global Constraints}: By restricting the admissible warping
        paths to specific regions within the cost matrix, such as the
        Sakoe-Chiba band or the Itakura parallelogram, computational
        efficiency can be improved and pathological alignments avoided.
  \item
        \textbf{Multiscale DTW}: This approach uses a hierarchical strategy,
        starting with a coarse alignment at a low resolution and then refining
        it at progressively higher resolutions, allowing for more efficient
        and robust alignment of long sequences.
\end{itemize}
\section{Applications}
A few applications of music synchronization:
\begin{itemize}
  \item
        \textbf{Multimodal Music Navigation}\cite{Damm2012}: Synchronization results can be
        used to create user interfaces that allow users to browse and explore
        music in a multimodal way. An example is the \textbf{Interpretation
          Switcher}, which enables seamless switching between different
        recordings of the same musical work.
  \item
        \textbf{Score Viewer Interface}\cite{inproceedings}: This interface provides a
        synchronized view of sheet music and an audio recording, highlighting
        the corresponding musical measures during playback.
  \item
        \textbf{Tempo Curves}: By aligning a performance to a reference score,
        it is possible to extract tempo curves that represent the relative
        tempo changes in the performance compared to the reference.
\end{itemize}
\section{In The End}
The chapter provides a comprehensive overview of the key
concepts, techniques, and applications of music synchronization,
highlighting the role of chroma features and DTW in
aligning different musical representations. It also emphasizes the
importance of understanding the various DTW variants and their impact on
the alignment process, and showcases the potential of music
synchronization for enhancing the user experience and facilitating music
analysis.
\section{Feedback}
\begin{itemize}
  \item This report helped me better understand the DTW variants, and I really liked the technique related to Multiscale DTW.
  \item I did not mention the online approach of DTW because did not understand it.
\end{itemize}
% %%%%%%%%%%%%%%%%%%%%%%%%%%%%%%%%%%%%%%%%%%%%%%%%%%%%%%%%%%%%%%%%%%%%%%%%%%%%%%
% \section{Main Section}
% \label{section:main}
% %%%%%%%%%%%%%%%%%%%%%%%%%%%%%%%%%%%%%%%%%%%%%%%%%%%%%%%%%%%%%%%%%%%%%%%%%%%%%%
% This chapter focuses on the process of music synchronization, which involves aligning different musical representations, such as audio recordings and sheet music. The core concept is to extract meaningful features from these representations and then employ a technique called Dynamic Time Warping (DTW) to find the optimal temporal correspondence between them.
% The text emphasizes the importance of chroma features in music synchronization. Chroma features essentially capture the distribution of energy across the 12 pitch classes (C, C\verb|#|, D, ..., B) while disregarding octave information. This characteristic makes them robust to variations in timbre and dynamics, allowing for effective comparison of different musical interpretations.
% The process of extracting chroma features starts with a log-frequency spectrogram. This spectrogram is generated by transforming an audio recording using the Short-Time Fourier Transform (STFT), which represents the audio signal's frequency content over time. The frequency axis of this spectrogram is then converted from Hertz to a logarithmic scale based on the pitches of the equal-tempered scale, resulting in the log-frequency spectrogram.
% From this log-frequency spectrogram, the chromagram is derived by summing up all pitch coefficients that belong to the same chroma. The resulting chroma features are robust to variations in timbre and instrumentation.
% The text further discusses various preprocessing techniques that can enhance the quality and effectiveness of chroma features, including:
% •	Logarithmic Compression: This step helps to balance out the large dynamic range of chroma features, enhancing the small but relevant values that may be obscured by larger values.
% •	Normalization: This process ensures that the chroma vectors are invariant to changes in dynamics, making them more comparable.
% Dynamic Time Warping (DTW) is then introduced as a technique for aligning two sequences of chroma features. DTW aims to find an optimal alignment that minimizes the total cost of matching corresponding elements in the sequences, effectively compensating for differences in tempo between different musical interpretations.
% A warping path represents the alignment between the two sequences, defined as a sequence of index pairs that satisfy specific conditions:
% •	Boundary condition: The path starts at the beginning of both sequences and ends at the end of both sequences.
% •	Monotonicity condition: The path progresses monotonically through both sequences, reflecting faithful timing.
% •	Step size condition: The path only allows for specific movements between cells of the cost matrix, ensuring a continuous alignment.
% The total cost of a warping path is calculated by summing up the local cost of aligning each pair of elements connected by the path. An optimal warping path is the one with the minimal total cost among all possible paths, and the DTW distance between two sequences is defined as the total cost of this optimal path.
% The text then details a dynamic programming algorithm for efficiently computing the DTW distance and the corresponding optimal warping path. The algorithm operates by recursively building an accumulated cost matrix, where each entry represents the minimum cost of aligning prefixes of the two input sequences.
% Several DTW variants are presented, each offering specific controls and optimizations for the alignment process:
% •	Modified Step Size Condition: By adjusting the allowed steps in the warping path, it is possible to constrain the local slope of the alignment, preventing overly stretched or compressed sections.
% •	Local Weights: Different weights can be assigned to vertical, horizontal, and diagonal steps in the warping path, allowing for preferences in the alignment direction.
% •	Global Constraints: By restricting the admissible warping paths to specific regions within the cost matrix, such as the Sakoe-Chiba band or the Itakura parallelogram, computational efficiency can be improved and pathological alignments avoided.
% •	Multiscale DTW: This approach uses a hierarchical strategy, starting with a coarse alignment at a low resolution and then refining it at progressively higher resolutions, allowing for more efficient and robust alignment of long sequences.
% The text concludes by discussing various applications of music synchronization techniques:
% •	Multimodal Music Navigation: Synchronization results can be used to create user interfaces that allow users to browse and explore music in a multimodal way. An example is the Interpretation Switcher, which enables seamless switching between different recordings of the same musical work.
% •	Score Viewer Interface: This interface provides a synchronized view of sheet music and an audio recording, highlighting the corresponding musical measures during playback.
% •	Tempo Curves: By aligning a performance to a reference score, it is possible to extract tempo curves that represent the relative tempo changes in the performance compared to the reference.
% In summary, the text provides a comprehensive overview of the key concepts, techniques, and applications of music synchronization, highlighting the role of chroma features and Dynamic Time Warping in aligning different musical representations. It also emphasizes the importance of understanding the various DTW variants and their impact on the alignment process, and showcases the potential of music synchronization for enhancing the user experience and facilitating music analysis.

% %%%%%%%%%%%%%%%%%%%%%%%%%%%%%%%%%%%%%%%%%%%%%%%%%%%%%%%%%%%%%%%%%%%%%%%%%%%%%%
% % Here you find some additional LaTex code fragments for including figures and
% % references which you may find helpful.
% %%%%%%%%%%%%%%%%%%%%%%%%%%%%%%%%%%%%%%%%%%%%%%%%%%%%%%%%%%%%%%%%%%%%%%%%%%%%%%

% Example for a citation:~\cite{Mueller21_FMP_SPRINGER}.

% Example for figure: Figure~\ref{figure:example}.

% \newpage

% %%%%%%%%%%%%%%%%%%%%%%%%%%%%%%%%%%%%%%%%%%%%%%%%%%%%%%%%%%%%%%%%%%%%%%%%%%%%%%
% \section{Feedback}
% \label{section:feedback}
% %%%%%%%%%%%%%%%%%%%%%%%%%%%%%%%%%%%%%%%%%%%%%%%%%%%%%%%%%%%%%%%%%%%%%%%%%%%%%%


% %-----------------------
% \begin{figure}[t]
%     \centering
%     \includegraphics[width=5cm]{figure_example.png}
%     \caption{Example for a figure.}
%     \label{figure:example}
% \end{figure}
% %-----------------------


% Feedback may also be given in form of bullet points
% \begin{itemize}
%     \item This was good....
%     \item I did not like...
% \end{itemize}



%%%%%%%%%%%%%%%%%%%%%%%%%%%%%%%%%%%%%%%%%%%%%%%%%%%%%%%%%%%%%%%%%%%%%%%%%%%%%%%%%%%%%%%%%%%%%%%%%%
\bibliographystyle{abbrv}
\small
\bibliography{references}
%%%%%%%%%%%%%%%%%%%%%%%%%%%%%%%%%%%%%%%%%%%%%%%%%%%%%%%%%%%%%%%%%%%%%%%%%%%%%%%%%%%%%%%%%%%%%%%%%%



\end{document}
